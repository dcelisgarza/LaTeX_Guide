\chapter{Float Environments}
%
Floats are the names \LaTeX~gives environments which are uncoupled
from normal text, i.e. can freely float around the document, namely
figures and tables.  \LaTeX~generally does a relatively good job at
placing floats where they make sense.  Unfortunately, it sometimes does
an atrocious one---especially in documents with highly customised
layouts---so the user has to intervene.  Fortunately, there exist
optional placement specifiers which give the user control over float
placement.  Table \ref{t:placspec}\footnote{Edited from
  \url{https://en.wikibooks.org/wiki/LaTeX/Floats,_Figures_and_Captions}}
shows possible placement specifiers and their meaning.
\begin{table}[!htbp]
  \centering
  \caption{Table with placement specifiers and their description.}
  \label{t:placspec}
  \begin{tabular}{cl}
    \toprule
    Specifier & Description\\
    \midrule
    h   & Place \emph{approximately} where it occurs in the source code.\\
    t   & Place at the \emph{top} of the page.\\
    b   & Place at the \emph{bottom} of the page.\\
    p   & Place in a \emph{dedicated} page.\\
    H   & Place \emph{exactly} where it occurs in the source code, needs \verb|float| package.\\
    !   & Overrides \LaTeX parameters for determining good float positions.\\
    !htbp   & Overrides \LaTeX parameters and finds the best position out of all placings.\\
    \bottomrule
  \end{tabular}
\end{table}

Along with these, there's the command, \verb|\FloatBarrier|, which
forces all previous floats to be rendered before it.  And the
\verb|placeins| package, often used with the \verb|section| option,
which forces floats to be placed within the section they are called.

Generally, floats require a caption, which is added by the command
\verb|\caption{}|.  As a rule, table captions in tables are found at
the top, while figure captions at the bottom.  There is also a starred
version, \verb|\caption*{}|, that eliminates the float number, useful
when tables or figures need notes.  In order to reference a float, it
must be assigned a label, \verb|\label{}|, after the caption.  If a
float has no caption but needs a label, an empty caption can be added
before labeling, \verb|\caption{}\label{label}|\footnote{Do \emph{not}
  use starred captions when labeling, otherwise there will be
  nothing---but emptiness and deep internal sadness---for the label to
  reference.}.  Floats are also generally centered on a page/column,
so the \verb|\centering| alignment switch is usually the first line in
a float environment.  In multicolumn documents, the normal
environments may overlap with the text and each other.  Starred
versions, \verb|{figure*}| and \verb|{table*}| avoid this issue by
reserving the whole page width (the value of \verb|\width|) for the
float.  Unfortunately, not all placement options work with starred
floats.
%
\subsection{Figures}
%
Documents often require the user to add images as figures or
standalone inputs.  As with many things, it is common knowledge that
an angel looses its wings every time someone inserts a \texttt{.jpg}
into a \LaTeX document.  So for the love of ceiling
cat \includegraphics[scale=0.4]{ceiling-cat.jpg}, use vectorised
images such as \texttt{.eps}, \texttt{.ps}, and \texttt{.pdf}.  The
only exceptions are scatter plots and heat maps, where it's advisable
to use \texttt{.bmp} or \texttt{.png} instead---as they are more
accurate in such cases.

The \verb|figure| environment is generally used in the following way:
\begin{verbatim}
	\begin{figure}[<placement_specifier>]
	    \centering
	    \includegraphics[options]{imagename.extension}
	    \caption{caption}
	    \label{f:label}
	\end{figure}
\end{verbatim}

\verb|\includegraphics[]{}| can be used on its own to include in-line
images.  Its options and their descriptions are found in table
\ref{t:inclgraph}\footnote{Edited from
  \url{https://en.wikibooks.org/wiki/LaTeX/Importing_Graphics}.}.
\begin{table}
  \centering
  \caption{Options and their descriptions for the \texttt{\textbackslash includegraphics} command.}
  \label{t:inclgraph}
  \begin{tabular}{rl}
    \toprule
    Option				& Description\\
    \midrule
    \verb|width=x|		& Image width $x$.\\
    \verb|height=y|		& Image height $y$.\\
    \verb|keepaspectratio|		& \verb|true|, \verb|false|. Keep aspect ratio with max width $x$ and height $y$.\\
    \verb|scale=x|		& Scale by factor $x$.\\
    \verb|angle=x|		& Rotate by $x$° (standard rotation).\\
    \verb|trim=l b r t|	& Trim left l, bott b, right r, top t (broken in \XeLaTeX).\\
    \verb|clip|			& Has to be \verb|true| for trim to work.\\
    \verb|resolution=x|	& Image resolution in $x$ dpi.\\
    \bottomrule
  \end{tabular}
\end{table}
\FloatBarrier
\subsection{Subfigures}
%
The \verb|subfigure| environment can only be declared within the
\verb|figure| environment, and requires the \verb|subcaption| package.
Its usage is as follows:
\begin{verbatim}
	\begin{figure}
	    \centering
	    \begin{subfigure}[t]{0.45\textwidth} % t = align subfigure on top.
	        \includegraphics[width=\textwidth]{dank.jpg}
	        \caption{A dank meme.}
	        \label{f:dankmeme}
	    \end{subfigure}
	    ~ % add desired spacing between images, e. g. ~, \quad, \qquad, \hfill etc.
	    % (or a blank line to move the subfigure to a next line.)
	    \begin{subfigure}[b]{0.45\textwidth} % align subfigure at the bottom.
	        \includegraphics[width=\textwidth]{ceiling-cat.jpg}
	        \caption{Ceiling cat.}
	        \label{f:ceilingcat}
	    \end{subfigure}

	    \begin{subfigure}[c]{0.45\textwidth} % b = align subfigure on its centre.
	        \includegraphics[width=\textwidth]{watoge.jpg}
	        \caption{Watoge}
	        \label{f:watoge}
	    \end{subfigure}
	    ~
	    \begin{subfigure}[c]{0.45\textwidth} % b = align subfigure on its centre.
	        \includegraphics[width=\textwidth]{MLG.png}
	        \caption{MLG.}
	        \label{f:mlg}
	    \end{subfigure}
	    \caption{Pictures of memes.}\label{f:memes}
	\end{figure}
\end{verbatim}
Which produces the output found in figs. \ref{f:memes},
\ref{f:dankmeme}, \ref{f:ceilingcat}, \ref{f:watoge} and \ref{f:mlg}:
\begin{figure}[!htbp]
    \centering
    \begin{subfigure}[t]{0.45\textwidth} % t = align subfigure on top.
    \includegraphics[width=\textwidth]{dank.jpg}
    \caption{A dank meme.}
    \label{f:dankmeme}
    \end{subfigure}
    ~ % add desired spacing between images, e. g. ~, \quad, \qquad, \hfill etc.
      % (or a blank line to move the subfigure to a next line.)
    \begin{subfigure}[b]{0.45\textwidth} % align subfigure at the bottom bottom.
        \includegraphics[width=\textwidth]{ceiling-cat.jpg}
        \caption{Ceiling cat.}
        \label{f:ceilingcat}
    \end{subfigure}

    \begin{subfigure}[c]{0.45\textwidth} % b = align subfigure at its centre.
        \includegraphics[width=\textwidth]{watoge.jpg}
        \caption{Watoge}
        \label{f:watoge}
    \end{subfigure}
     ~
    \begin{subfigure}[c]{0.45\textwidth} % t = align subfigure on its centre.
        \includegraphics[width=\textwidth]{mlg.png}
        \caption{MLG.}
        \label{f:mlg}
    \end{subfigure}
    \caption{Pictures of memes.}\label{f:memes}
\end{figure}
\FloatBarrier
%
\section{Tables}
%
Tables in \LaTeX~can either be the source of great pride or great
pain.  Done right, they look classy and slick\ldots done wrong, they
are tacky and I hate them\footnote{Stickin' it to `The Man'.}.  Before
designing a table, you must answer me these questions
three\footnote{You seek the holy grail.}:
\begin{enumerate}
\item Readability: is it easily understandable?
\item Minimalism: does it contain the minimum required
  information---does it minimise redundancy?
\item Simplicity: how many borders does it have---does it really need
  all those vertical and/or horizontal borders?
\end{enumerate}
The package \verb|booktabs| provides improved versions of the
horizontal line commands, \verb|\hline| and
\verb|\cline{}|.\footnote{If there's a need for non-tabular horizontal
  lines---such as redefining footnote rules---you should still use
  \texttt{\textbackslash hline}.}  They are: \verb|\toprule|,
\verb|\midrule|, \verb|\cmidrule{}|, and \verb|\bottomrule|.  They all
provide additional spacing to make tables more readable, and accept
options to modify their trimming and thickness.  A table is declared
in the following way:
\begin{verbatim}
	\begin{table}[<placement_specifier>]
	    \centering
	    \caption{caption}
	    \label{t:label}
	    \begin{tabular}{<n columns (with alignment), vertical borders>}
	        \toprule
	        heading 1 & ... & heading n
	        \midrule
	        a_11 & ... & a_1n \\
	        .
	                .
	                        .
	        a_m1 & ... & a_mn \\
	        \bottomrule
	    end{tabular}
	\end{table}
\end{verbatim}

In line with the three principles of table design, one must avoid
borders as much as possible without sacrificing readability.  For most
purposes, tables only need horizontal lines at the top, bottom, and
for separating heading from the body.  There is also
\verb|\cmidrule{i-j}| which spans columns $i$ to $j$.  The
\verb|array| package extends table functionality by providing extra
arguments for table customisation.  The columns are represented by
\verb|l|, \verb|c| and \verb|r| for left, center and right alignment,
they can be placed immediately after one another or separated by
spaces.  Vertical borders are denoted by \verb+|+ between columns.
The actual columns are then separated by ampersands \verb|&|.  If the
number of columns is $n$, the number of ampersands is $n-1$.

To showcase what we have learned of tables up to now, table
\ref{t:example} is an example of a very poorly formatted table.  Its
generating code is:
\begin{verbatim}
	\begin{table}
	    \centering
	    \caption{Poorly designed and formatted table.}
	    \label{t:example}
	    \begin{tabular}{|cl|||r||}
	        \cline{2-3}
	        Measurement & Magnitude & Units \\
	        \hline
	        Speed		& 15		& m/s \\
	        Time		& $\infty$	& s \\
	        \hline
	        \hline
	    \end{tabular}
	\end{table}
\end{verbatim}
\begin{table}[!htbp]
    \centering
    \caption{Poorly designed and formatted table.}
    \label{t:example}
    \begin{tabular}{|cl|||r||}
        \cline{2-3}
        Measurement & Magnitude & Units \\
        \hline
        Speed		& 15		& m/s \\
        Time		& $\infty$	& s \\
        \hline
        \hline
    \end{tabular}
\end{table}

The documentation for \verb|booktabs| contains further information on
table design, formatting, and package features.
%
\subsection{Multicolumn and Multirow Tables}
%
Sometimes tables have fields which span more than one row or column
(sometimes both).  When this is unavoidable, the \verb|multirow|
package provides the commands:
\begin{verbatim}
	\verb|\multirow{num_rows}{width}{contents}|
	\verb|\multicolumn{num_columns}{column_alignment and borders}{contents}|
\end{verbatim}
An asterisk ``*'' as width in \verb|\multirow| stands for the
contents' natural width.  \verb|\multicolumn| gobbles all relevant
columns, so there's no need to add the gobbled ampersands.  If
\verb|\multirow| is used, all subsequent rows in the same column(s)
must be left empty and take up the same number of columns.  Table
\ref{t:multirowcol} was sourced directly from the wiki entry for
tables\footnote{\url{https://en.wikibooks.org/wiki/LaTeX/Tables}}
because it's an excellent non-trivial minimal working example:
\begin{verbatim}
	\begin{table}
	    \centering
	    \caption{Multirow and multicolumn table.}
	    \label{t:multirowcol}
	    \begin{tabular}{cc|c|c|c|c|l}
	        \cline{3-6}
	        & & \multicolumn{4}{ c| }{Primes} \\ \cline{3-6}
	        & & 2 & 3 & 5 & 7 \\ \cline{1-6}
	        \multicolumn{1}{ |c  }{\multirow{2}{*}{Powers} } &
	        \multicolumn{1}{ |c| }{504} & 3 & 2 & 0 & 1 &     \\ \cline{2-6}
	        \multicolumn{1}{ |c  }{}                        &
	        \multicolumn{1}{ |c| }{540} & 2 & 3 & 1 & 0 &     \\ \cline{1-6}
	        \multicolumn{1}{ |c  }{\multirow{2}{*}{Powers} } &
	        \multicolumn{1}{ |c| }{gcd} & 2 & 2 & 0 & 0 & min \\ \cline{2-6}
	        \multicolumn{1}{ |c  }{}                        &
	        \multicolumn{1}{ |c| }{lcm} & 3 & 3 & 1 & 1 & max \\ \cline{1-6}
	    \end{tabular}
	\end{table}
\end{verbatim}
\begin{table}[!htbp]
  \centering
  \caption{Multirow and multicolumn table.}
  \label{t:multirowcol}
  \begin{tabular}{cc|c|c|c|c|l}
    \cline{3-6}
    & & \multicolumn{4}{ c| }{Primes} \\ \cline{3-6}
    & & 2 & 3 & 5 & 7 \\ \cline{1-6}
    \multicolumn{1}{ |c  }{\multirow{2}{*}{Powers} } &
    \multicolumn{1}{ |c| }{504} & 3 & 2 & 0 & 1 &     \\ \cline{2-6}
    \multicolumn{1}{ |c  }{}                        &
    \multicolumn{1}{ |c| }{540} & 2 & 3 & 1 & 0 &     \\ \cline{1-6}
    \multicolumn{1}{ |c  }{\multirow{2}{*}{Powers} } &
    \multicolumn{1}{ |c| }{gcd} & 2 & 2 & 0 & 0 & min \\ \cline{2-6}
    \multicolumn{1}{ |c  }{}                        &
    \multicolumn{1}{ |c| }{lcm} & 3 & 3 & 1 & 1 & max \\ \cline{1-6}
    \end{tabular}
\end{table}
%
\subsection{Multipage Tables}
%
Sometimes a project requires tables which span more than a single
page.  For cases such as these, the \verb|longtable| package provides
the \verb|longtable| environment, which offers customisation commands
relevant to multipage
tables\footnote{\url{http://mirrors.rit.edu/CTAN/macros/latex/required/tools/longtable.pdf}}.
\verb|longtable| does not need the \verb|tabular| environment, it is
declared as follows:
\begin{verbatim}
	\begin{longtable}{<n columns (with alignment), vertical borders>}
	    \caption{caption}
	    \label{t:longtable}
	    \toprule
	    heading 1 & ... & heading n
	    \midrule
	    a_11 & ... & a_1n \\
	    .
	            .
	                    .
	    a_m1 & ... & a_mn \\
	    \bottomrule
	\end{longtable}
\end{verbatim}
\verb|longtable| also works with the functionality offered by the
\verb|multirow| package.  However, multiple compilations ($4+$) may be
required to get long tables with multirows and/or multicolumns looking
as they should.
%
