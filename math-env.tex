\chapter{Mathematical Environments}
%
The primary reason why Don Knuth started developing \TeX~was to create an easy and powerful tool for typesetting academic mathematical documents. There are absurdly many mathematical tools and symbols in \LaTeX, but this document will only touch the most common ones non-mathematicians use. The packages \verb|mathtools|, \verb|bm|, \verb|amssymb|, and \verb|pxfonts| respectively provide mathematical packages and symbols, corrected bold maths fonts, other mathematical symbols, and more fonts and symbols. Mathematical symbols, functions and typefaces cannot exist outside a mathematical environment, though some have non-maths---or text---versions, usually \verb|text<symbol_name>| or \verb|\text<command_name>{}|.
%
\section{Maths and Display Maths}
%
The two most basic mathematical environments are the inline and display maths environments. The inline maths environment is delimited by single dollar signs \verb|$ inline maths $|, which yields $ inline maths $. Inline maths has its limitations because it is text-bound, which can cause problems when using mathematical notation that extends beyond the bounds of a normal line of text. For example $ \frac{\sum\limits_{i=0}^{n} i}{\sum\limits_{i=0}^{n} 2i} = \frac{1}{2} $ generated with,
\begin{verbatim}
	\frac{\sum\limits_{i=0}^{n} i}{\sum\limits_{i=0}^{n} 2i} = \frac{1}{2}
\end{verbatim}
Which pushes line spacing beyond acceptable levels. In cases such as these, when equation numbering is not an issue, one can use the display maths environment delimited by double dollar signs, \verb|$$ display maths $$|, or the preferred method (better spacing) of using backslash followed by chevrons, \verb|\[ display maths \]|. Display maths uncouples the notation from the text and places it \emph{exactly} where the command is found the point it's called from. It also lets large symbols scale accordingly. Using the previous equation as an example, \[ \frac{\sum\limits_{i=0}^{n} i}{\sum\limits_{i=0}^{n} 2i} = \frac{1}{2}\,, \] it's easy to see the need for both. It's also important to note that all mathematical environments gobble spaces. Use \verb|~|, \verb|\,|, or any other spacing command to add spaces. If you need text to be printed within a mathematical environments use the \verb|\text{}| command and make sure to explicitly add spaces at the beginning and end of the normal text to ensure proper spacing. For example: \verb|$x\textrm{apples}+y\text{pears}$|, $x\textrm{apples}+y\text{pears}$ vs. \verb|$x\textrm{ apples }+y\text{ pears}$|, $x\textrm{ apples }+y\text{ pears}$.
%
\section{Common Symbols and Functions}
%
There are a myriad of symbols and functions for virtually \emph{all} areas of maths. This document touches on the ones non-mathematicians will generally use. If anyone is curious about more symbols make sure to check out Scott Pakin's 164-page list of \LaTeX~symbols \cite{symbols}.
%
\subsection{Greek Maths Symbols}
%
All maths greek letters are backslash commands of the letter's name. There are capitalised versions which are written the same, but start with a capital letter. Some letters also have variations which are denoted as \verb|var<letter_name>|. Table \ref{t:mg} contains an example of this.
\begin{table}[!htbp]
    \centering
    \caption{Examples of maths greek letters.}
    \label{t:mg}
    \begin{tabular}{ccc}
        $\phi$ & $\varphi$ & $\Phi$\\
        \verb|\phi| & \verb|\varphi| & \verb|\Phi|
    \end{tabular}
\end{table}
There are also bold versions of some symbols, which is specified by the \verb|\boldsymbol{}| command. When a bold symbol does not exist there's the \verb|\pmb{}| (poor man's bold) which prints slightly offset symbols to give the appearance of bold typeface.
%
\subsection{Common Binary Operations}
%
\begin{table}[!htbp]
    \centering
    \caption{Common binary operations.}
    \label{t:bo}
    \begin{tabular}{ccccccc}
        $+$ &   $-$   & $\pm$   &   $\mp$   &   $\times$    &   $\div$  &   $\cdot$ \\
        \verb|+|    &   \verb|-|    &   \verb|\pm|  &   \verb|\mp|  &   \verb|\times|   &   \verb|\div| &   \verb|\cdot|    \\
    \end{tabular}
\end{table}
%
\subsection{Common Relation Symbols}
%
\begin{table}[!htbp]
    \centering
    \caption{Common relation symbols.}
    \label{t:rs}
    \begin{tabular}{ccccccccc}
        $=$ & $\neq$ & $\equiv$ & $\simeq$ & $\approx$ & $\sim$ & $\cong$ & $<$ & $>$\\
        \verb|=| & \verb|\neq| & \verb|\equiv| & \verb|\simeq| & \verb|\approx| & \verb|\sim| & \verb|\cong| & \verb|<| & \verb|>| \\
        $\leq$ & $\geq$ & $\ll$ & $\gg$ & $\in$ & $\ni$ & $\propto$ & $\perp$ & $\parallel$\\
        \verb|\leq| & \verb|\geq| & \verb|\ll| & \verb|\gg| & \verb|\in| & \verb|\ni| & \verb|\propto| & \verb|\perp| & \verb|\paralel|\\
    \end{tabular}
\end{table}
%
\subsection{Arrows}
%
Arrows follow a few rules, capitalising the first letter makes the arrow have a two line shaft. The direction is specified first, if the arrow has two heads, left goes before right and up before down. There are two standard types, arrows (always stated as singular) and harpoons (singular except left right harpoons). Harpoons can only be horizontal and the orientation of the spike is specified last. Table \ref{t:arr} contains some examples. Arrows can also be specified as long, which goes first in the command.
\begin{table}[!htbp]
    \centering
    \caption{Arrows.}
    \label{t:arr}
    \begin{tabular}{ccc}
        $\Leftrightarrow$ & $\updownarrow$ & $\leftrightharpoons$ \\
        \verb|\Leftrightarrow| & \verb|\updownarrow| & \verb|\leftrightharpoons| \\
        $\leftharpoonup$ & $\Longleftrightarrow$	& $\longrightarrow$\\
        \verb|\leftharpoonup| & \verb|\Longleftrightarrow| & \verb|\longrightarrow|
    \end{tabular}
\end{table}
%
\subsection{Common Miscellaneous Symbols}
%
\begin{table}[!htbp]
    \centering
    \caption{Common miscellaneous symbols.}
    \label{t:miscfym}
    \begin{tabular}{ccccccc}
        $\ldots$ & $\cdots$ & $\vdots$ & $\ddots$ & $\infty$ & $\hbar$ & $\emptyset$\\
        \verb|\ldots| & \verb|\cdots| & \verb|\vdots| & \verb|\ddots| & \verb|\infty| & \verb|\hbar| & \verb|\emptyset|\\
        $\exists$ & $\nabla$ & $\neg$ & $\Re$ & $\Im$ & $\angle$ & $\partial$\\
        \verb|\exists| & \verb|\nabla| & \verb|\neg| & \verb|\Re| & \verb|\Im| & \verb|\angle| & \verb|\partial|\\
        & \multicolumn{5}{c}{$\therefore$} & \\
        & \multicolumn{5}{c}{\texttt{\textbackslash therefore}} & \\
    \end{tabular}
\end{table}
%
\subsection{Common Functions}
%
\LaTeX~has commands for all basic functions. They are backslash commands defined as the function's shortened name without abuse of notation, e.g. inverse trigonometric functions are written as \verb|\arc<function>|. In fact, most functions are defined exactly how they are written in rigorous pen \& paper maths.
%
\subsection{Common Variable-Sized Symbols}
%
Some symbols can change size depending on where they are in-document. Sometimes however, they're not big enough for their argument. This is solved by the nest-able \verb|\mathlarger{}| command, provided by the \verb|relsize| package. The number of integrals is defined by the number of $i$'s in the command---up to 4 for normal integrals, 3 for cyclic ones\footnote{Triple cyclic integrals need the \texttt{pxfonts} or \texttt{txfonts} package}. Linear integrals can also have dots. 
\begin{table}[!htbp]
    \centering
    \caption{Common variable-sized symbols.}
    \label{t:vss}
    \begin{tabular}{cccccccc}
        $\sum$ & $\prod$ & $\int$ & $\oint$ & $\idotsint$ & $\oiiint$ & $\frac{a}{b}$ & $\dfrac{a}{b}$ \\
        \verb|\sum| & \verb|\prod| & \verb|\int| & \verb|\oint| & \verb|\idotsint| & \verb|\oiiint| & \verb|\frac{a}{b}| & \verb|\dfrac{a}{b}| \\
    \end{tabular}
\end{table}
%
\subsection{Non-trivial Delimiters}
%
\begin{table}[!htbp]
    \centering
    \caption{Non-trivial delimiters.}
    \label{t:ntd}
    \begin{tabular}{cccccccc}
        $\|$ & $|$ & $\langle$ & $\rangle$ & $\lfloor$ & $\rfloor$ & $\lceil$ & $\rceil$ \\
        \verb+\|+ & \verb+|+ & \verb|\langle| & \verb|\rangle| & \verb|\lfloor| & \verb|\rfloor| & \verb|\lceil| & \verb|\rceil| \\
    \end{tabular}
\end{table}
%
\subsection{Common Mathematical Accents}
%
There are many mathematical accents, however we will only mention the most common ones used in science. The majority of the most common ones are provided by the \AmSTeX~packages. The \verb|mathdots| package redefines the spacing between dots for three and four dots. The number of dots is defined y the number of $d$'s in the \verb|\dots| command. There also exist dotless versions of $\imath$ and $\jmath$, provided by \verb|\imath| and \verb|\jmath| respectively---useful for properly adding hats to orthogonal unit vectors.

Some accents also have wide versions. Some bar accents (including arrows) are denoted as
\verb|\over<bar_accent>|, while other wide accents are denoted as \verb|\wide<accent>|. Table \ref{t:ma} contains some common mathematical accents and their respective commands.
\begin{table}[!htbp]
    \centering
    \caption{Common mathematical accents.}
    \label{t:ma}
    \begin{tabular}{ccccccc}
        $\hat{a}$ & $\check{a}$ & $\bar{a}$ & $\vec{a}$ & $\dot{a}$ & $\ddot{a}$ & $\tilde{a}$ \\
        \verb|\hat{a}| & \verb|\check{a}| & \verb|\bar{a}| & \verb|\vec{a}| & \verb|\dot{a}| & \verb|\ddot{a}| & \verb|\tilde{a}| \\
        \multicolumn{2}{c}{$\mathring{a}$} & \multicolumn{2}{c}{$\widehat{abc}$} & $\sqrt{abc}$ & \multicolumn{2}{c}{$\sqrt[n]{abc}$} \\
        \multicolumn{2}{c}{\texttt{\textbackslash mathring\{a\}}} & \multicolumn{2}{c}{\texttt{\textbackslash widehat{abc}}} & \verb|\sqrt{abc}| & \multicolumn{2}{c}{\texttt{\textbackslash sqrt[n]{abc}}} \\
        \multicolumn{3}{c}{$\underline{abc}$} & \multicolumn{4}{c}{$\overbrace{\underbrace{abc}}$} \\
        \multicolumn{3}{c}{\texttt{\textbackslash underline\{abc\}}} & \multicolumn{4}{c}{\texttt{\textbackslash overbrace\{\textbackslash underbrace\{abc\}\}}} \\
        & \multicolumn{5}{c}{$\overleftarrow{\underleftrightarrow{abc}}$} & \\
        & \multicolumn{5}{c}{\texttt{\textbackslash overleftarrow\{\textbackslash underleftrightarrow\{abc\}\}}} & \\
    \end{tabular}
\end{table}

Furthermore, \verb|\overbrace{}| and \verb|\underbrace{}| may have sub/superscripts automatically placed directly below and above them without resorting to \verb|\limits_{}^{}|. For example:
\begin{verbatim}
	z = \overbrace{\underbrace{x}_{\text{Real}} + 
	    i \underbrace{y}_{\text{Imaginary}}}^{\text{Complex number.}}
\end{verbatim}
\[z = \overbrace{\underbrace{x}_{\text{Real}} + i \underbrace{y}_{\text{Imaginary}}}^{\text{Complex number.}}\]
\LaTeX~will let the sub and superscripts take spacing priority, if this is a problem, placing the whole sub/superscript inside the \verb|\mathclap{}| command will eliminate the extra spacing.
%
\subsection{Limits and Substack Commands}
%
Some symbols may have arguments and/or sub/supersripts in different places. Placing the \verb|\limits| command immediately after a symbol, moves the immediately following sub/superscript below and above it, respectively---having both at the same time is not a requirement. The \verb|\substack| command allows subscripts to be stacked. Table \ref{t:limits} shows this.
\begin{table}[!htbp]
    \centering
    \caption{Effect of the \texttt{\textbackslash limits} command.}
    \label{t:limits}
    \begin{tabular}{rcl}
        \toprule
        No \verb|\limits| & \verb|\limits| & Command\\
        \midrule
        $\sum_{i}$ & $\sum\limits_{i}$ & \verb|$\sum\limits_{i}$| \\[0.5cm]
        $\sum^{j}$ & $\sum\limits^{j}$ & \verb|$\sum\limits^{j}$| \\[0.5cm]
        $\sum_{i}^{j}$ & $\sum\limits_{i}^{j}$ & \verb|$\sum\limits_{i}^{j}$| \\[0.5cm]
        $\sum_{\substack{1\le i\le n\\ i\ne j}}$ & $\sum\limits_{\substack{1\le i\le n\\ i\ne j}}$ & \verb|\sum\limits_{\substack{1\le i\le n\\ i\ne j}}|\\[0.1cm]
        \bottomrule
    \end{tabular}
\end{table}
%
\subsection{Scaled Delimiters}
%
It's very common to have large equations which require large delimiters, these are obtained by adding \verb|\left|, \verb|\right| and \verb|\middle| (optional) before the actual delimiter, they also scale the delimiter according to the size of the equation. It's important to remember that braces still need a backslash for \LaTeX~to print them, \verb|\left\{ \right\}|. Sometimes the delimiter is unilateral (only one is printed), but their scaled versions must be closed (i.e. have both left and right components), in such cases a single full stop, \verb|.| becomes the closing delimiter. The following equations are good examples of this.
\begin{verbatim}
	\begin{align}
	    &\exp(-\sum\limits_{i}^{N}\frac{E_{i}}{k_{b}T}) \\
	    &\exp\left(-\sum\limits_{i}^{N}\frac{E_{i}}{k_{b}T}\right) \\
	    &\int\limits_{a}^{b} \exp\left(-\frac{E}{k_{b}T}\right)\,\mathrm{d}E 
	    = -k_{b}T\exp\left(-\frac{E}{k_{b}T}\right)|_{a}^{b} \\
	    &\int\limits_{a}^{b} \exp\left(-\frac{E}{k_{b}T}\right)\,\mathrm{d}E 
	    = \left.-k_{b}T\exp\left(-\frac{E}{k_{b}T}\right)\right|_{a}^{b}
	\end{align}
\end{verbatim}
\begin{align}
    &\exp(-\sum\limits_{i}^{N}\frac{E_{i}}{k_{b}T}) \\
    &\exp\left(-\sum\limits_{i}^{N}\frac{E_{i}}{k_{b}T}\right) \\
    &\int\limits_{a}^{b} \exp\left(-\frac{E}{k_{b}T}\right)\,\mathrm{d}E 
    = -k_{b}T\exp\left(-\frac{E}{k_{b}T}\right)|_{a}^{b} \\
    &\int\limits_{a}^{b} \exp\left(-\frac{E}{k_{b}T}\right)\,\mathrm{d}E = \left.-k_{b}T\exp\left(-\frac{E}{k_{b}T}\right)\right|_{a}^{b}
\end{align}
%
\subsection{Common Maths Typefaces}
%
Oftentimes, different mathematical objects have specific notations. For most applications no packages are needed, but loading both \verb|bm| and \verb|mathtools| packages cover most other cases. Table \ref{t:mt} provides some examples.
\begin{table}[!htbp]
    \centering
    \caption{Common mathematical typefaces.}
    \label{t:mt}
    \begin{tabular}{rl}
        \toprule
        Command & Result \\
        \midrule
        \verb|\mathtt{ABCdef123}| & $\mathtt{ABCdef123}$ \\
        \verb|\mathbb{ABCdef123}| & $\mathbb{ABCdef123}$ \\
        \verb|\mathbf{ABCdef123}| & $\mathbf{ABCdef123}$ \\
        \verb|\mathcal{ABCdef123}| & $\mathcal{ABCdef123}$ \\
        \verb|\mathsf{ABCdef123}| & $\mathsf{ABCdef123}$ \\
        \verb|\mathit{ABCdef123}| & $\mathit{ABCdef123}$ \\
        \verb|\mathrm{ABCdef123}| & $\mathrm{ABCdef123}$ \\
        \verb|\mathfrak{ABCdef123}| & $\mathfrak{ABCdef123}$ \\
        \verb|\bm{ABCdef123}| & $\bm{ABCdef123}$ \\
        \bottomrule
    \end{tabular}
\end{table}
%
\section{Align}
%
The \verb|align| environment is the one you should be using for all your equation typesetting needs. It does everything the \verb|equation| and \verb|eqnarray| packages do, but better. The starred variant removes all equation numbers. Its behaviour is exactly the same as \verb|tabular|, but does not require the user to declare the number of columns or alignment within them. It also automatically defines all the lines it encloses as a mathematical environment, removing the need to add dollar signs. Indiviual equations can be labeled by placing \verb|\label{<label>}| before a line break, and referenced with \verb|\eqref{}|. Individual equation numbers can be removed with the \verb|\nonumber| command before a line break. The following equations are aligned on the equal sign:
\begin{verbatim}
	\begin{align}
	    F(x) &= \int f(x)\,\mathrm{d}x + \mathrm{C}\\
	    f(x) &= \frac{\mathrm{d}F(x)}{\mathrm{d}x} \nonumber\\
	    g(x,y,z) &= 0
	\end{align}
\end{verbatim}
\begin{align}
    F(x) &= \int f(x)\,\mathrm{d}x + \mathrm{C}\\
    f(x) &= \frac{\mathrm{d}F(x)}{\mathrm{d}x} \nonumber\\
    g(x,y,z) &= 0
\end{align}
%
\subsection{Subequations}
%
Subequations are declared by the environment of the same name, which wraps around other equation environments such as, \verb|align|, \verb|eqnarray| and \verb|equation|. They have same behaviour as normal equations, including labeling and referencing. Their numbering can also be customised via \verb|\renewcommand{}{}|. Using the aforementioned example within the \verb|subequations| environment yields:
\begin{verbatim}
	\begin{subequations}
	    \begin{align}
	        F(x) &= \int f(x)\,\mathrm{d}x + \mathrm{C} \label{e:int}\\
	        f(x) &= \frac{\mathrm{d}F(x)}{\mathrm{d}x} \nonumber\\
	        g(x,y,z) &= 0
	    \end{align}
	\end{subequations}
\end{verbatim}
\begin{subequations}
	\begin{align}
        F(x) &= \int f(x)\,\mathrm{d}x + \mathrm{C} \label{e:int}\\
        f(x) &= \frac{\mathrm{d}F(x)}{\mathrm{d}x} \nonumber\\
        g(x,y,z) &= 0
    \end{align}
\end{subequations}
%
\section{Matrices}
%
Matrices can be represented in many ways depending on the context, or to denote certain matrix operations. Pre-defined matrices require the \verb|mathtools| package. The default types are shown in table \ref{t:mat}, which is edited from the mathematics entry in the \LaTeX~wikibook\footnote{\url{https://en.wikibooks.org/wiki/LaTeX/Mathematics}}. Non-starred versions center the columns by default, starred variants allow user-defined alignments. The matrix environment must be declared within a maths environment. Matrices can also be created using the \verb|array| environment, which is essentially the maths-mode version of the \verb|tabular| environment.
\begin{table}[!htbp]
    \centering
    \caption{Matrix types.}
    \label{t:mat}
    \begin{tabular}{cc}
        \toprule
        Command & Delimiter \\
        \midrule
        \verb|pmatrix| & (\quad) \\
        \verb|bmatrix| & [\quad] \\
        \verb|Bmatrix| & \{\quad\} \\
        \verb|vmatrix| & |\quad| \\
        \verb|Vmatrix| & $\|\quad\|$ \\
        \bottomrule
    \end{tabular}
\end{table}
%
\subsection{Aligned and Array}
%
There are times when annotating equations is useful for the reader, or desired by the author. In cases such as these, the \verb|aligned| provides a quick solution, though it only provides a single ampersand for aligning items. In cases where more than one ampersand is needed use \verb|array| instead, which is the maths equivalent of \verb|tabular|.\footnote{They are used and declared in exactly the same way; only \texttt{tabular} is used within normal text, and \texttt{array} within maths environments.}
\begin{verbatim}
	\begin{align}
	    \left.
	        \begin{aligned}
	            f(x,y,z) &= \alpha x + \beta y + \gamma z + \delta \\
	            f(x,y) &= \alpha x + \beta y + \gamma \\
	            f(x) &= \alpha x + \beta
	        \end{aligned}
	    \right\}\text{Linear equations.}
	\end{align}
\end{verbatim}
\begin{align}
    \left.
    \begin{aligned}
        f(x,y,z) &= \alpha x + \beta y + \gamma z + \delta \\
        f(x,y) &= \alpha x + \beta y + \gamma\\
        f(x) &= \alpha x + \beta
    \end{aligned}
    \right\}\text{Linear equations.}
\end{align}
%
\section{Cases}
%
The \verb|cases| environment allows the user to define piecewise functions without the pain of recurring to \verb|\aligned| and delimiter scaling. Similarly to the \verb|eqnarray| and \verb|aligned| environments, it provides a single ampersand for aligning. Take eq. \eqref{e:abs} for example. There also exists \verb|dcases|, which provides a display version of the environment---it properly scales symbols---as shown in eqs. \eqref{e:case} and \eqref{e:dcases}. It also has a starred version shown in Eq. \eqref{e:dcases*}, which sets everything to the right of the ampersand to normal text---therefore all maths in this region must be within single dollar signs.
\begin{verbatim}
\begin{align}
    |f(x) + b| + c & =  \begin{cases}
                            -f(x) - b + c & \textrm{if } x < 0 \\
                            f(x) + b + c & \textrm{otherwise}
                        \end{cases} \label{e:abs}\\
    g(x) & =    \begin{cases}
                    \int f(x)\,\mathrm{d}x & \textrm{if } x < 0\\
                    0 & \textrm{otherwise}
                \end{cases} \label{e:case} \\
    g(x) & =    \begin{dcases}
                    \int f(x)\,\mathrm{d}x & \textrm{if } x < 0\\
	                0 & \textrm{otherwise}
                \end{dcases} \label{e:dcases} \\
    g(x) & =    \begin{dcases*}
                    \int f(x)\,\mathrm{d}x & if $x < 0$\\
                    0 & otherwise
                \end{dcases*} \label{e:dcases*}             
	\end{align}
\end{verbatim}
\begin{align}
    |f(x) + b| + c & =  \begin{cases}
                            -f(x) - b + c & \textrm{if } x < 0 \\
                            f(x) + b + c & \textrm{otherwise}
                        \end{cases} \label{e:abs}\\
    g(x) & =    \begin{cases}
                    \int f(x)\,\mathrm{d}x & \textrm{if } x < 0\\
                    0 & \textrm{otherwise}
                \end{cases} \label{e:case} \\
    g(x) & =    \begin{dcases}
                    \int f(x)\,\mathrm{d}x & \textrm{if } x < 0\\
                    0 & \textrm{otherwise}
                \end{dcases} \label{e:dcases} \\
    g(x) & =    \begin{dcases*}
                    \int f(x)\,\mathrm{d}x & if $x < 0$\\
                    0 & otherwise
                \end{dcases*} \label{e:dcases*}             
\end{align}
%