\chapter{Referencing and Citing}
\section{Referencing}
%
Referencing in \LaTeX~is extremely easy to do, all it requires is a
label---\verb|\label{}|---and reference---\verb|ref{}| for
non-equations, and \verb|\eqref{}| for equations---commands.
Referencing floats also requires the label to come after a non-starred
caption.

As long as an item can be \emph{uniquely} identified, it can be
labeled and referenced.  Labels are placed immediately following
sectioning, ToC, appendix, and indexing commands; before line breaks
in equations; and after the caption in floats.  The \pkg{hyperref}
package loads further referencing capability, such as hyperlinks and
colour-coded links depending on hyperlink type.

The package \pkg{cleveref} does not work on Overleaf, but adds
incredibly useful functionality such as smart references, reference
formats, reference ranges, compressed references, capitalisation, etc.
It doesn't work when \pkg{amsmath} and \pkg{hyperref} are
simultaneously loaded.  Fortunately \pkg{mathtools} is vastly superior
to \pkg{amsmath} in every respect, and does \emph{not} present this
bug; so use it in place of \pkg{amsmath} and enjoy the wonders of
\pkg{cleveref}.
%
\section{Bibliography}
%
One of the appeals of \LaTeX~is bibliography management.  There are
two ways of doing so:
\begin{itemize}
    \item Manually: better known as the hard way.
    \item \BibTeX: better known as the only way.  Google scholar also
      provides \texttt{.bib} files by clicking Cite under an article
      and selecting \BibTeX~at the bottom left of the popup window.
\end{itemize}
They can be referenced using the \verb|\cite[]{}|, where the optional
chevrons can be omitted, and the bibliographic label is placed inside
the braces.
%
\subsection{Manual}
%
Doing things manually means the user has to hard-code the style into
the bibliography.  The user adds the \verb|thebibliography|
environment, which takes a numeric argument for the maximum number of
bibliographic items:
\begin{verbatim}
	\begin{thebibliography}{99}
	    \bibitem{<biblabel1_>} ...
	    \bibitem{<biblabel2_>} ...
	    ...
	    \bibitem{<biblabel_n>}
	\end{thebibliography}
\end{verbatim}
%
\subsection{\BibTeX}\label{sb:bib}
%
The preferred way of handling bibliographies is by using the
\pkg{natbib} package together with \BibTeX.  \pkg{natbib} offers
further citing options and commands which are too numerous and readily
available\footnote{\url{https://en.wikibooks.org/wiki/LaTeX/More_Bibliographies}}
to be placed here.  Though I'd like to mention \verb|\nocite{}|,
because it adds the specified entry to the bibliography, but doesn't
print the reference in-text.  If the argument is an asterisk, *, all
entries in the \verb|.bib| file are added to the bibliography
regardless of whether they were cited or not.

Using \BibTeX~requires a separate \verb|.bib| file, where entries are
defined according to specific formats depending on the type of
document or file to be cited, the first string of which is the
bibliographic label.  The following commands are then placed after all
the document's content and before \verb|\end{document}|:
\begin{verbatim}
    \bibliographystyle{<style>}
    \bibliography{<filename>}
\end{verbatim}
where the \verb|<style>| is replaced by a bibliographic style, and
\verb|<filename>| is the \verb|.bib| file's name without extension
(also accepts relative and absolute paths).  There are many
bibliographic styles found in \pkg{natbib}'s manual or its wikibook
entry\footnote{\url{https://en.wikibooks.org/wiki/LaTeX/Bibliography_Management\#Natbib}}.
They are not added here because their nuances are too subtle and
involved for the purposes of this document.  Other
packages---especially journal-specific packages---add specialist
styles.

In order to fully compile a file with a \BibTeX~bibliography, the
following compilation train must be followed:
\begin{enumerate}
    \item \XeLaTeX (or engine of your choice).
    \item \BibTeX.
    \item \XeLaTeX (or engine of your choice).
    \item \XeLaTeX (or engine of your choice).
\end{enumerate}
This will ensure all references are parsed and the bibliography will
be updated to show all the right entries with the correct style.  Some
editors can automate this process in a single shortcut.
%
