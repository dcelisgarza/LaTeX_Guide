\chapter{Introduction to \LaTeX}
%
\section{Quick Description}\label{s:qd}
\LaTeX~(written \verb|\LaTeX|) is an open-source document typesetting
language originally created as \TeX~by renowned mathematician and
computer scientist, Donald E. Knuth, as a means of incorporating the
best typesetting practices of academic---predominantly
mathematical---journals, in an easy to use package.

\TeX~has now been expanded into \LaTeX---which wraps around many
primitives and makes the language more accessible. What started as an
academic typesetting language for mathematicians has expanded its
scope to include all three basic sciences, computer science,
engineering, writing, and even music.  It's an extremely flexible tool
that can produce all kinds of documents from personal letters to
beamer presentations, and everything in between.

The language is characterised by its modularity, flexibility, and
ability to produce beautiful, standardised, and reproducible
documents.  Admittedly, the learning curve can be very
steep---especially if one is pressed for time---but the end result is
\emph{very} worth it.

\LaTeX~is not a word processor, it's a typesetting languange.  As
such, every \LaTeX~document is comprised of two large sections:
\begin{enumerate}
\item Preamble: Where the user loads relevant modules,
  \emph{packages}, which define a document's characteristics and
  provide functionality via \emph{backslash commands} (commands for
  short) and \emph{environments}.
\item Body: Contains the document's actual content.
\end{enumerate}
%
\section{Getting Started}
%
There are several freely available distributions depending on your OS,
as shown in table \ref{t:dist}.
\begin{table}[!htbp]
  \centering
  \caption{\LaTeX~distributions.}
  \label{t:dist}
  \begin{tabular}{cc}
    \toprule
    Distribution & OS \\
    \midrule
    \TeX~Live & Linux \\
    Mac\TeX & OS X \\
    Mik\TeX~\& Pro\TeX t & Windows\\
    \bottomrule
  \end{tabular}
\end{table}
Since \LaTeX~is a language rather than a word processor, documents can
be produced in any plain text editor and compilation carried out via
console command.  However, this is inefficient and there are better
alternatives:
\begin{itemize}
\item Vim\LaTeX: a Vim module for writing \LaTeX~documents.
\item Emacs/XEmacs + AUC\TeX~+ Ref\TeX: modules for writing
  \LaTeX~documents on Emacs/XEmacs.
\item Dedicated editor: there are a variety of dedicated editors, some
  of which come with different distributions.  \TeX works for Mik\TeX,
  \TeX Shop for Mac\TeX~and \TeX Studio for Pro\TeX t.  My personal
  favourite is \TeX Studio, because it's
  \begin{inparaenum}[\itshape 1\upshape)]
  \item lightweight,
  \item easy to use,
  \item compatible with Windows, Linux and OS X, and
  \item comes with a lot of quality of life and advanced features;
    such as automatic quotes, user-defined compilation trains,
    wizards, macros, folding, etc.
  \end{inparaenum}
\end{itemize}

Being a non-interpreted programming language, \LaTeX~codes generate
documents upon successful compilation.  Compiling is done via console
commands or keyboard shortcuts.  All editors provide their own
defaults, but they're also re-configurable.  There is also something
known as a \emph{compilation engine}, of which there are a few.  The
two most modern are \XeLaTeX~(or \XeTeX), and \LuaLaTeX~(or \LuaTeX).
The standard is still PDF\LaTeX~which does not provide full unicode
encoding.  All the distributions mentioned above come with all three
(and more) engines.  The default compilation engine can be specified
within your editor of choice.  This document was created with
\XeLaTeX, because it is more mature (less buggy, more features) than
\LuaLaTeX, and more competent than PDF\LaTeX.  Which means you have to
use \XeLaTeX~if you want to compile the source code.

A file with a \verb|.bib| bibliography must also be run through the
\BibTeX~engine, which also comes in all aforementioned distributions
and can be done within your editor of choice.  Further details
provided in subsection \ref{sb:bib}.
%
\section{Best Practices}\label{s:bestpract}
%
Code can get ugly pretty quickly, especially when done wrong.  So here
is a checklist of things that will make your life easier.
\begin{itemize}
\item Indent sensibly. \LaTeX~does not interpret manual indents as
  spaces, that's what the spacing commands are for.
\item Break apart troublesome lines.  A single line break will not
  interrupt text flow unless a blank line is added between segments.
  This helps a lot when debugging and writing long equations.
\item Divide and conquer.  Use information theory ($\log_{2} x$) to
  optimise your debugging.  Divide things progressively in half
  (comment out segments), narrowing down your search by halving your
  search space with each iteration.
\item Structure your approach to equation writing.  Simplify equations
  as much as possible.  Typeset small, independent sections and
  incorporate them where they are needed \emph{after} they have been
  checked for mistakes.  Repeat process until the complete equation
  has been typeset.
\item Use sensible labels.  I like to use ``p''--part, ``c''--chapter,
  ``s''--section, ``sb''--subsection, ``ssb''--subsubsection,
  ``f''--figure, ``sf''--subfigure, ``t''--table, ``e''--equation,
  ``se''--subequations.  Followed by a colon and a short but
  descriptive name.
\item Automate as much as possible.  Define user commands for things
  you will use often.
\item Customise your editor of choice.  Take time to set up your
  theme, shortcuts, custom macros, auto-complete, compilation train,
  flags, dictionary, etc.  Make sure you learn your most used
  shortcuts, macros, and commands.
\end{itemize}
%
\section{Special Writing Characters}
%
There exist special writing characters which have specific,
non-trivial \LaTeX~characters.  Table \ref{t:swc} contains some of the
most common ones, their \LaTeX~character, and usage.
\begin{table}[!htbp]
  \centering
  \caption{Special writing characters.}
  \label{t:swc}
  \begin{tabular}{rccl}
    \toprule
    Name & Print & Code & Usage \\
    \midrule
    English Half-Quotation & `' & \verb|`'| & Nested quotation. \\
    English Quotation & ``'' & \verb|``''| & Speech or technical term. \\
    Hyphen & - & \verb|-| & Compound words. \\
    En Dash & -- & \verb|--| & Ranges, scores, conflict/connection. \\
    Em Dash & --- & \verb|---| & Cursory explanation, no pre/post-space. \\
    Double Em Dash & ------ & \verb|------| & Interrupted word, no pre-space. \\
    Ellipses & \ldots & \verb|\ldots| & Pause, pre/post-space optional (format). \\
    \bottomrule
  \end{tabular}
\end{table}
%
\section{Reserved \LaTeX~Characters}\label{s:rlc}
%
Like many programming languages, \LaTeX~contains certain reserved
characters used for different native processes.  Table \ref{t:reschar}
shows the symbol, its escape method (how to print them), and function.
\begin{table}[!htbp]
  \centering
  \caption{Table of special \LaTeX~characters, their escape method, and function.}
  \label{t:reschar}
  \begin{tabular}{rcl}
    \toprule
    Symbol	&	Escape	&	Function\\
    \midrule
    \% &	\verb|\%|	&	Comment.\\
    \textbackslash &	\verb|\textbackslash|	&	Escape other characters, initiate macro. \\
    \& &	\verb|\&|	&	Align delimiter.\\
    \~{} &	\verb|\~{}| or \verb|$ \sim $|	&	Non-breaking space. It's a character, not a space.\\
    \# &	\verb|\#|	&	Macro numbered argument.\\
    \$ &	\verb|\$|	&	Maths environment (\verb|$ maths $|).\\
    \$\$ &	\verb|\$\$|	&	Display maths environment (\verb|$$ maths $$|).\\
    \char`\^ &	\verb|\char`\^|	&	Superscript (maths environment).\\
    \_ &	\verb|\_|	&	Subscript (maths environment).\\
    \textbackslash\textbackslash &	\verb|\textbackslash|($ \times 2$)	&	Line break.\\
    \{\} &	\verb|\{\}|	&	Argument.\\
    \bottomrule
  \end{tabular}
\end{table}
