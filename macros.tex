%-------User-defined commands-------%
% Note: For chemistry commands use mchem package commands.
\newcommand{\footref}[1]{\textsuperscript{\ref{#1}}}	  % Footnotes in case the normal footnote command
							  % takes a hike. Type \footref{<argument>} to use.
\newcommand{\dif}[2]{\frac{\textrm{d} #1}{\textrm{d} #2}} % Total derivative. Type \dif{a}{b} for the
							  % total derivative of a with respect to b.
\newcommand{\dpar}[2]{\frac{\partial #1}{\partial #2}}    % Partial derivative. Type \dpar{a}{b} for the
							  % partial derivative of a with respect to b.
\newcommand{\nnum}{\nonumber\\}		% Nonumber and skip line. Useful for very long equations. Add the command
					% \nnum at the end of equations that don't fit in a single line.

\usepackage{xparse}

\ExplSyntaxOn

% Based on http://tex.stackexchange.com/a/263556/17423
\cs_new_protected:Nn \dcelisgarza_ltxguide_make_equiv:Nn
  { \NewDocumentCommand #1 {} {#2} }

\NewDocumentCommand \NewMarkup { m }
  { \keyval_parse:NNn \use_none:n \dcelisgarza_ltxguide_make_equiv:Nn {#1} }

\NewMarkup
  {
    \XeLaTeX = \hologo{XeLaTeX},
    \XeTeX = \hologo{XeTeX},
    \LuaLaTeX = \hologo{LuaLaTeX},
    \LuaTeX = \hologo{LuaTeX},
    \BibTeX = \hologo{BibTeX},
    \AmSTeX = \hologo{AmSTeX},
    \pkg = \textsf,
    \cls = \textsf,
    \opt = \texttt,
    \env = \texttt,
    \cs  = \texttt,
    \marg =,
    \oarg =,
    \meta =,
    \sub =,
%    \sup =,
  }

\ExplSyntaxOff