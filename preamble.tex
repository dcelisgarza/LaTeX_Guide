%-----------------------------------------IMPORTANT-----------------------------------------%
% PDFLaTeX is outdated and bad. Use XeLaTeX for compiling. In contrast to PDFLaTeX, 
% XeLaTeX natively supports unicode characters (accents, degree symbol, emoji, dongers), 
% has more features, and many more available packages. You may also need a few specialty 
% packages that may not come as default in your TeX installation. Such packages are: 
% xcolor, minted, mhchem and cleveref.

% Comment out or remove any packages you do not use. Uncomment packages you do.

% \vfill and \hfill commands respectively fill the page vertically and horizontally so that
% it is completely occupied by its content. No need to fiddle with \vspace{} or \hspace{}.
%-------------------------------------------------------------------------------------------%

\documentclass[12pt]{report}      % Document type declaration.
\sloppy                           % LaTeX becomes less strict with formatting

%--------------Logos--------------%
\usepackage{doc}		  % \BibTeX, \AmSTeX
\usepackage{metalogo}		  % \XeLaTeX, \XeTeX, \LuaLaTeX, \LuaTeX

%------------Packages-------------%
\usepackage[usenames,dvipsnames,  % Fancy colours.
            svgnames,table,
            x11names,
            hyperref]{xcolor}
\usepackage{graphicx}             % Image manipulation.
\usepackage{subcaption}           % Subfigures.
\usepackage[colorlinks=true,      % Hyperlinks
            pdfstartview=FitV,
            linkcolor=blue,
            citecolor=blue,
            urlcolor=blue,
            hyperfootnotes=true
            ]{hyperref}         
\usepackage[version=3]{mhchem}     % Chemistry formulae, reaction equations, arrow customisation
				   % and notation (different arrow types, catalysis, annotations, etc).
\usepackage[comma,square,          % Citing style.
           numbers,
           sort&compress
           ]{natbib}
\newcommand\bmmax{2}			   % Load more maths alphabets.
\usepackage{bm,mathtools,amssymb,pxfonts}  % Maths packages. {bold maths, (loads the amsmath 
					   % package with extra environments, symbols, and fonts),
					   % more symbols, more symbols}.
\usepackage{paralist}              % In-paragraph lists \begin{inparaenum}[options] \end{imparaenum}.
\usepackage[top=1in,               % Document's geometric specs.
		   bottom=1in,
		   left=1.5in,
		   right=1in
		   ]{geometry}
\usepackage{booktabs}		   % Professional-looking tables \toprule, \midrule, \cmidrule{}, \bottomrule
\usepackage{multirow}              % Multirow and multicolumn tables (avoid when possible). If you cannot 
				   % avoid such monsters use the Excel2LaTeX package (for Excel) from ctan 
				   % or Calc2LaTeX (for LibreOffice Calc) from sourceforge.
\usepackage{chngcntr}		   % Managing and creating counters (variables which count things).
\usepackage[toc,page]{appendix}    % Appendix in table of contents.
%\usepackage{minted}                % Displaying code colourfully. Requires a python 2.7 or higher installation for pygments. Also add "-shell-escape" flag to the compilation command.
\usepackage{pdfpages}              % Include pdf pages.
\usepackage[section]{placeins}     % Place floats in the section they're called.
\usepackage{verbatim}              % Text editing features.
\usepackage{array}		   % Extends the functionality of tables.
\usepackage{lscape}                % Use landscape pages.
\usepackage{exscale}               % Automatically scale mathematical notation appropriately. 
				   % Useful when using special notation such as sum and 
				   % multiplication notation, or nested sub and superscripts. 
				   % No need to use \displaystyle.
\usepackage{cleveref}              % Clever cross-references. Extremely useful package. 
				   % Automatically states whether a reference is to a table, 
				   % figure, equation, or bibliography item. Has a lot of 
				   % options such as ranges, and citation styles.
\usepackage{lmodern}		   % Cleaner, modern version of Computer Modern fonts.
%\allowdisplaybreaks               % Lets the align environment ignore page breaks.
%\newfontfamily\ubuntumono{Ubuntu Mono}  % For the theorists out there who want to use Ubuntu for your
					 % write up and "experiments", and want to be fancy in including
					 % shell commands, untuntu features or workflows, etc.
\graphicspath{{images/}}          % Relative image path. Keep your stuff organised.

%---------Special options----------%
%\counterwithout{figure}{chapter}  % Uncomment to remove chapter number from figures.
%\counterwithout{table}{chapter}   % Uncomment to remove chapter number in tables.
\setlength{\topskip}{0in}          % Top paragraph spacing.
\setlength{\parskip}{2ex}          % Bottom paragraph spacing.
\linespread{1}			   % Inter-line spacing.
\renewcommand{\footnoterule}{      % Footnote rule specs.
	\kern -5.4pt
	\hrule width \textwidth height 0.4pt
	\kern 5pt
	}
\makeatletter
\newcommand*{\codefont}{	   % Font size for code. Activate by typing \codefont.
	\@setfontsize
	\codefont{8.1pt}{8.1pt}
	}
\makeatother

%-------User-define commands-------%
% Note: For chemistry commands use mchem package commands.
\newcommand{\footref}[1]{\textsuperscript{\ref{#1}}}	  % Footnotes in case the normal footnote command 
							  % takes a hike. Type \footref{<argument>} to use.
\newcommand{\dif}[2]{\frac{\textrm{d} #1}{\textrm{d} #2}} % Total derivative. Type \dif{a}{b} for the 
							  % total derivative of a with respect to b.
\newcommand{\dpar}[2]{\frac{\partial #1}{\partial #2}}    % Partial derivative. Type \dpar{a}{b} for the 
							  % partial derivative of a with respect to b.
\newcommand{\nnum}{\nonumber\\}		% Nonumber and skip line. Useful for very long equations. Add the command
					% \nnum at the end of equations that don't fit in a single line.
\newcommand{\spth}{./scripts/}		% Relative path for scripts.
\newcommand{\cpth}{./code/}		% Relative path for codes.

%----------Page Numbering----------%
\newcounter{originalpagenumber}    % Page number counter for the bibliography after appendices.