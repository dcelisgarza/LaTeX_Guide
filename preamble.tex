%-----------------------------------------IMPORTANT-----------------------------------------%
% PDFLaTeX is outdated and bad. Use XeLaTeX for compiling. In contrast to PDFLaTeX, 
% XeLaTeX natively supports unicode characters (accents, degree symbol, emoji, dongers), 
% has more features, and many more available packages. You may also need a few specialty 
% packages that may not come as default in your TeX installation. Such packages are: 
% xcolor, minted, mhchem and cleveref.

% Comment out or remove any packages you do not use. Uncomment packages you do.

% \vfill and \hfill commands respectively fill the page vertically and horizontally so that
% it is completely occupied by its content. No need to fiddle with \vspace{} or \hspace{}.
%-------------------------------------------------------------------------------------------%

\documentclass[12pt]{report}      % Document type declaration.

%--------------Logos--------------%
\usepackage{doc}		  % \BibTeX, \AmSTeX
\usepackage{metalogo}		  % \XeLaTeX, \XeTeX, \LuaLaTeX, \LuaTeX

%------------Packages-------------%
\usepackage[hyperref]{xcolor}     % Fancy colours.
\usepackage{graphicx}             % Image manipulation.
\usepackage{subcaption}           % Subfigures.
\usepackage[colorlinks=true,      % Hyperlinks
            pdfstartview=FitV,
            linkcolor=blue,
            citecolor=blue,
            urlcolor=blue,
            hyperfootnotes=true
            ]{hyperref}         
\usepackage[version=3]{mhchem}     % Chemistry formulae, reaction equations, arrow customisation
				   % and notation (different arrow types, catalysis, annotations, etc).
\usepackage[comma,square,          % Citing style.
           numbers,
           sort&compress
           ]{natbib}
\newcommand\bmmax{2}			   % Load more maths alphabets.
\usepackage{bm,mathtools,amssymb,pxfonts}  % Maths packages. {bold maths, (loads the amsmath 
					   % package with extra environments, symbols, and fonts),
					   % more symbols, more symbols}.
\usepackage{paralist}              % In-paragraph lists \begin{inparaenum}[options] \end{imparaenum}.
\usepackage[
  top    = 1in,
  bottom = 1in,
  left   = 1.5in,
  right  = 1in
]{geometry}                        % Control page layout
\usepackage{booktabs}		   % Professional-looking tables \toprule, \midrule, \cmidrule{}, \bottomrule
\usepackage{multirow}              % Multirow and multicolumn tables (avoid when possible). If you cannot 
				   % avoid such monsters use the Excel2LaTeX package (for Excel) from ctan 
				   % or Calc2LaTeX (for LibreOffice Calc) from sourceforge.
\usepackage[toc,page]{appendix}    % Appendix in table of contents.
\usepackage[section]{placeins}     % Place floats in the section they're called.
\usepackage{verbatim}              % Text editing features.
\usepackage{cleveref}              % Clever cross-references. Extremely useful package. 
				   % Automatically states whether a reference is to a table, 
				   % figure, equation, or bibliography item. Has a lot of 
				   % options such as ranges, and citation styles.
\graphicspath{{images/}}          % Relative image path. Keep your stuff organised.

%---------Special options----------%
%\counterwithout{figure}{chapter}  % Uncomment to remove chapter number from figures.
%\counterwithout{table}{chapter}   % Uncomment to remove chapter number in tables.
\setlength{\topskip}{0in}          % Top paragraph spacing.
\setlength{\parskip}{2ex}          % Bottom paragraph spacing.
\linespread{1}			   % Inter-line spacing.

%-------User-define commands-------%
% Note: For chemistry commands use mchem package commands.
\newcommand{\footref}[1]{\textsuperscript{\ref{#1}}}	  % Footnotes in case the normal footnote command 
							  % takes a hike. Type \footref{<argument>} to use.
\newcommand{\dif}[2]{\frac{\textrm{d} #1}{\textrm{d} #2}} % Total derivative. Type \dif{a}{b} for the 
							  % total derivative of a with respect to b.
\newcommand{\dpar}[2]{\frac{\partial #1}{\partial #2}}    % Partial derivative. Type \dpar{a}{b} for the 
							  % partial derivative of a with respect to b.
\newcommand{\nnum}{\nonumber\\}		% Nonumber and skip line. Useful for very long equations. Add the command
					% \nnum at the end of equations that don't fit in a single line.

%----------Page Numbering----------%
\newcounter{originalpagenumber}    % Page number counter for the bibliography after appendices.
%
% Packages of potential interest:
%
% - minted   :: Displaying code colourfully.  Requires Python 2.7 or
%               higher for Pygments.  Also add "-shell-escape" flag to
%               the compilation command.
%
%               (Pure-TeX alternative: listings)
% - lscape   :: Easily specify specific pages to be landscape.
% - chngcntr :: Managing 'counters' (figure 1, table 2.3, etc.).
% - lmodern  :: Cleaner, more complete version of 'Computer Modern'.
%               It doesn't make sense to use this *and* fontspec.
% - pdfpages :: Include pages of a PDF.
% - fontspec :: Use any font installed on your system.  Refer to docs.
%
% For all LaTeX packages and document classes, use the Texdoc program.
% For example, to view the 'mhchem' documentation,
%
%     texdoc mhchem
%
% This is how you can find additional loading options, functionality,
% and answers to common questions about the package/class.  It's a
% good idea to skim (at least) the documentation of every
% class/package you use.
%
%
% Commands/switches of potential interest:
%
% \allowdisplaybreaks :: (amsmath)
%                        Allow the {align} environment break across pages.
