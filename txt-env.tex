\chapter{Non-Maths Environments}
%
There are many environments that respect the text flow and are not
specifically designed for mathematical or special notation.  This
section contains descriptions for some of the most common ones.
%
\section{List Structures}
%
Both stratified listing environments have four standard nesting levels
(\verb|enum<x>| counters).  Bullet shapes, enumeration characters, and
list levels can be specified and customised in the environment's
options,
\verb|\begin{environment}[options]|\footnote{\url{https://en.wikibooks.org/wiki/LaTeX/List_Structures}}.
  The \verb|enumitem| package expands this capability.
%
\subsection{Itemize}
%
The \verb|itemize| environment lists items as bullet-point lists.
Bullet points can be redefined with \verb|\renewcommand{}{}| or with
functionality offered by the \verb|enumitem| package.  The environment
is used as follows:
\begin{verbatim}
	\begin{itemize}
    	\item Nesting level 1.
	    \begin{itemize}
    	    \item Nesting level 2.
        	\begin{itemize}
            	\item Nesting level 3.
	            \begin{itemize}
    	            \item Nesting level 4.
        	    \end{itemize}
	        \end{itemize}
	    \end{itemize}
	\end{itemize}
\end{verbatim}
And outputs:
\begin{itemize}
\item Nesting level 1.
  \begin{itemize}
  \item Nesting level 2.
    \begin{itemize}
    \item Nesting level 3.
      \begin{itemize}
      \item Nesting level 4.
      \end{itemize}
    \end{itemize}
  \end{itemize}
\end{itemize}
Printed indentations are a result of the environment, not code
indentation (see section \ref{s:contents}).
%
\subsection{Enumerate Environment}
%
The \verb|enumerate| environment places items as numbered lists.  The
numbering can be redefined with \verb|\renewcommand{}{}| or with
functionality offered by the \verb|enumitem| package.  It's declared
as follows:
\begin{verbatim}
	\begin{enumerate}
    	\item Nesting level 1.
	    \begin{enumerate}
    	    \item Nesting level 2.
        	\begin{enumerate}
            	\item Nesting level 3.
	            \begin{enumerate}
    	            \item Nesting level 4.
        	        \item Nesting level 4.
	            \end{enumerate}
    	        \item Nesting level 3.
        	\end{enumerate}
	        \item Nesting level 2.
    	\end{enumerate}
	    \item Nesting level 1.
	\end{enumerate}
\end{verbatim}
And outputs:
\begin{enumerate}
\item Nesting level 1.
  \begin{enumerate}
  \item Nesting level 2.
    \begin{enumerate}
    \item Nesting level 3.
      \begin{enumerate}
      \item Nesting level 4.
      \item Nesting level 4.
      \end{enumerate}
    \item Nesting level 3.
    \end{enumerate}
  \item Nesting level 2.
  \end{enumerate}
\item Nesting level 1.
\end{enumerate}
Printed indentations are a result of the environment, not code
indentation (see section \ref{s:contents}).
%
\subsection{In-line Lists}
%
There are a couple of ways to do this, with either the \verb|paralist|
or \verb|enumitem| package.  For \verb|paralist|, the counter tokens
are A, a, I, i and 1; and respectively represent the \verb|\Alph|,
\verb|\alph|, \verb|\Roman|, \verb|\roman|, and \verb|\arabic| counter
styles.  Conversely, \verb|enumitem| uses them explicitly.  Both
environments can take optional font shape switches and commands, as
well as other symbols for item label customisation.  In
\verb|paraenum| this is done like so:
\begin{verbatim}
	\begin{inparaenum}[\itshape A\upshape)]
	    \item Item 1.
	    \item Item 2.
	\end{inparaenum}
\end{verbatim}
in \verb|enumitem| the way to do this is:
\begin{verbatim}
	\begin{enumitem*}[label=\itshape\Alph*\upshape)]
	    \item Item 1.
	    \item Item 2.
	\end{enumitem*}
\end{verbatim}
They both produce:
\begin{inparaenum}[\itshape A\upshape)]
\item Item 1.
\item Item 2.
\end{inparaenum}
%
\section{Verbatim Environments}
%
Say you want to make a template, package, o \LaTeX~guide, so you need
to textually print \LaTeX~commands so people can actually read them.
Or you simply want to comment large sections of code.  Then the
\verb|verbatim| package is essential.
%
\subsection{Verbatim and Alltt}
%
The \verb|verbatim| environment \emph{explicitly} prints whatever is
written within it (spaces included), regardless of what it is.  It's
useful for printing \LaTeX~ commands in independent lines.  There is
also a way to print in-line verbatim, \verb+\verb|content|+, where the
first character after
\verb|\verb| is the delimiter---which can be any character except for *.  The \verb|alltt|
environment is somewhat different in that it accepts \emph{some}
commands within
it\footnote{\url{https://en.wikibooks.org/wiki/LaTeX/Paragraph_Formatting}}.
%
\subsection{Comment Environment}
%
The \verb|comment| environment is self-explanatory.  Anything written
within it will be treated as a comment and will therefore not be in
the output file.
%
